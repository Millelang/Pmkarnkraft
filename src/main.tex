\documentclass[11p]{article}
% Packages
\usepackage{amsmath}
\usepackage{graphicx}
\usepackage[swedish]{babel}
\usepackage[
    backend=biber,
    style=authoryear-ibid,
    sorting=ynt
]{biblatex}
\usepackage[utf8]{inputenc}
\usepackage[T1]{fontenc}
%Källor
\addbibresource{mall.bib}
\graphicspath{ {./images/} }

\title{PMmall \\ \small Fysik 1}
\author{Magnus Silverdal }
\date{\today}

\begin{document}

    \begin{titlepage}
        \begin{center}
            \vspace*{1cm}

            \Huge
            \textbf{Title}

            \vspace{0.5cm}
            \LARGE
            Subtitle

            \vspace{1.5cm}

            \textbf{Milton Långström}

            \vfill

            Ett PM om energiförsörjning \\
            Fysik 1

            \vspace{0.8cm}

            \includegraphics[width=0.4\textwidth]{NTI Gymnasiet_Symbol_print_svart}

            \Large
            Teknikprogrammet\\
            NTI Gymnasiet\\
            Umeå\\
            \today

        \end{center}
    \end{titlepage}
% Om arbetet är långt har det en innehållsförteckning, annars kan den utelämnas
    \tableofcontents
    \newpage

    \section{Inledning}
    Beskriv varför detta ämne är intressant eller viktigt. Vad är syftet med texten?
    Kärnkraft är en utav dagens mest diskuterade energikällor, många gillar det medans andra ogillar det.

    Syftet med detta pm är att läsaren ska få en bättre förståelse kring kärnkraft
    \subsection{frågeställningar}
    rada upp dina frågor i punktform
    \begin{enumerate}
        \item Hur fungerar ett kärnkraftverk?
        \item Vilka miljöpåverkan har ett kärnkraftverk lokalt och globalt?
        \item Hur påverkar kärnkraft samhället (Ekonomi/politik/konflikter/m.m.) lokalt och globalt?
    \end{enumerate}

    \section{Resultat}
    Här kommer allt med massor av mer rubriker och underrubriker
    \subsection{Så fungerar kärnkraftverk}
    Ett kärnkraftverk skapar elektricitet genom att man kokar massiva mängder vatten vilket i sin tur gör så att en turbin som är kopplad till en generator roterar.
    Vattnet inuti reaktorn kokas genom att man klyver atomer vilket sätter igång en kedjereaktion som värmer vattnet, denna kedjereaktion kallas för en fissionkedja.
    De atomer som klyvs i kärnkraftverk är oftast uran-235 eftersom att det är det grundämne som lättast att klyva, uran-235 är även ett väldigt radioaktivt ämne.

    \subsection{Globala miljökonsekvenser av kärnkraft}
    \subsection{Lokal miljöpåverkan av ett kärnkraftverk}
    \subsection{Såhär påverkar ett kärnkraftverk naturen}
    \subsection{}

    \section{Slutsatser}
    Här kan du dra slutsatser eller sammanfatta ditt resultat

% Mer saker som du kan ha nytta av.

    \section{Referenser}
    Referenser i text kan skrivas på två sätt: Enligt \textcite{Jens} kan man använde två typer av referenser, inbäddade i texten eller efter ett fakta \parencite{Fraenkel}. Ett till test för att se hur det ser ut \parencite[sid 55]{fermi}.

    \section{Annat som kan vara bra att veta}
    Om du vill ha kodstil och få med alla tecken kan du använda verbatim. då kan du skriva \verb|abcd!"#| utan problem...

    Citat skrivs mellan de konstiga symbolerna \verb|``| och \verb|''| för att de ska se bra ut ``se bra ut!''.
    \subsection{En underrubrik}
    \subsubsection{En underunderrubrik}
    \subsection{Ekvationer}
    Det är lätt att skriva matematik i \LaTeX

    Ekvation (\ref{grav}) känner ni igen...

    \subsection{figurer}
    Bilder placeras enklast på detta sätt. placeringen bestämmer \LaTeX och vi kan bara föreslå (h)är, (t)opp eller (b)otten. Ett utropstecken före tvingar lite mer men inte absolut. I bild \ref{varg} visas en varg
    \begin{figure}[!h]
        \includegraphics[width=0.8\textwidth]{../images/accelerationTime.png}
        \caption{Acceleration-tid diagram. Källa: Impuls Fysik 1}
        \label{varg}
    \end{figure}
    \printbibliography

\end{document}
